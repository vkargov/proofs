\documentclass{article}

\usepackage{tikz}

\title{Binet's formula for the Fibonacci sequence}

\begin{document}

\maketitle

\section{Introduction}
Okay, so this is just my take on Binet's formula that derives the closed-form Fibonacci number expression. This pretty much copies the proof given on page 30 of ``Elementary number theory in nine chapters'' by J. Tattersall, except that I try to justify the intuition why those steps were taken because I had some trouble doing it on my own.

\section{Proof}

\subsection{Golden ratio}

To begin, let us remind ourselves of how the golden ratio is defined.

Suppose we have a line:


\begin{tikzpicture}
  \draw (0,0) -- (4,0);
  \draw (0,-0.1) -- (0,0.1);
  \draw (3,-0.1) -- (3,0.1);
  \draw (4,-0.1) -- (4,0.1);
  \node at (1.5,0.3) {a};
  \node at (3.5,0.3) {b};
\end{tikzpicture}

\bigskip

$\frac{(a + b)}{a} = \frac{a}{b} = \phi$

Multiplying by $\phi$ we get the following quadratic equation:
$\phi^2 = \phi + 1$

Solving it with respect to $\phi$ gives us the golden ratio $\phi = \frac{(1 + \sqrt{5})}{2}$ and its conjugate $\hat{\phi} = \frac{(1 - \sqrt{5})}{2}$.

If we further multiply this quadratic expression by $\phi^n$ we will get
$$\phi^{n + 2} = \phi^{n + 1} + \phi^{n}$$
And likewise, the same will hold for the conjugate:
$$\hat{\phi}^{n + 2} = \hat{\phi}^{n + 1} + \hat{\phi}^{n}$$

\subsection{The great reveal}

Reminds you of something?

Like, maybe that Fibonacci sequence we wanted to represent with the golden ratio? $$F_{n+2} = F_{n+1} + F_{n}$$

\textbf{I wonder if $F$ can be represented as a linear combination of $\phi$ and $\hat{\phi}$... \textit{wink wink nudge nudge}}

$$F_n = C_1 \phi^n + C_2 \hat\phi^n$$

Well, since $F_1 = 1$ and $F_2 = 2$, we can just substitute the numbers and solve the system easily.

Result?

$$C_1 = \frac{1}{\sqrt{5}}, C_2 = -\frac{1}{\sqrt{5}}$$
thus
\begin{equation}
  F_n = \frac {\phi^n - \hat{\phi}^n}{\sqrt{5}}
\end{equation}

\section{Consecutive quotient limit}
As a bonus, consider the following limit:
$$\lim_{n\to\infty} \frac{F_{n+1}}{F_{n}}$$
By using the formula above we get:
$$\lim_{n\to\infty} \frac{F_{n+1}}{F_{n}} = \lim_{n\to\infty} \frac{\phi^{n+1} + \hat\phi^{n+1}}{\phi^{n} + \hat\phi^{n}} = \lim_{n\to\infty} \frac{\phi + \hat\phi (\frac{\hat\phi}{\phi})^n}{1 + (\frac{\hat\phi}{\phi})^n}$$
and since $$\lim_{n\to\infty}(\frac{\hat\phi}{\phi})^n = \lim_{n\to\infty}(\frac{1-\sqrt{5}}{1+\sqrt{5}})^n = 0 $$
we finally get
\begin{equation}
  \lim_{n\to\infty}\frac{F_{n+1}}{F_n} = \phi
\end{equation}
In other words, the ratio of two consecutive Fibonacci numbers approaches the golden ratio as the numbers grow larger!
This was first shown by Kepler back in the XVII century.

\begin{thebibliography}{1}

\bibitem{number_theory} James Tattersall {\em Elementary number theory in nine chapters} page 30.
\bibitem{clr} Cormen, Leiserson, Rivest {\em Introduction to algorithms} p.60, but they invite the reader to prove it by induction which again isn't very conducive to getting that ``aha!'' moment.

\end{thebibliography}

\end{document}